\section{The Current Scenario}
\label{sec:-current}
On the $12^{th}$ of March 2016, history was made. Lee Sedol, a professional Go Player who some consider to be the best player of the last decade, lost a game of Go. His third consecutive loss in a best-of-five competition. His opponent? AlphaGo an Artificial Intelligence based Go playing computer system\cite{alphagolee}. Although this wouldn't be considered anything new, with computers outsmarting their human opponents from Watson, IBM's Artificial Intelligence besting previous champions to win Jeopardy\cite{watsonjeopardy} to Deep Blue(another of IBM's attempts at Artificial Intelligence), the first computer to beat a Grand Master at Chess. But why was this such a historic victory? It was for two primary reasons; firstly the game of Go has about the same number of possible positions as there are atoms in the universe (this rules out simply computing every possible combination) and secondly, this is a milestone in the trend of using computers to replicate mental processes that were earlier considered to be the sole dominion of the living beings, particularly humans.

As mentioned earlier, we live in the Information Age, and changes have a tendency to propagate through vast sections of society fairly quickly. Globalization and the subsequent transfers of technology enabled by the Internet, allows for a proliferation of ideas and techniques that would have been unheard of a few decades ago. Automation's strongest argument has been the one to reduce manual labor and improve the quality of human lives. Although now, it would seem that this is no longer the case. Many proponents of automation have pointed to the migration of people from unskilled and semi-skilled labor to more skilled labor which rely on mental processes that could not have been automated at the time. It is evident that this sacred boundary is under threat of being or has already been, depending on who you ask, infringed upon.

The advent of Applications that have improved accessibility to services, from planning your week to simply setting an alarm or a reminder to do your taxes, all these tasks have been automated already. The automation of these activities begs the question whether this would make humans obsolete in performing these activities. The public sentiment does not seem to reflect this. In a study carried out by the \textit{Pew Research Center}\cite{workforcepew}, although about two-thirds of Americans believe that computers and robots will be performing most of the work being done by humans in the next 50 years, nearly 80 \% believe that their current jobs will still be around then.
	
A likely answer to this is skewed figure is probably to do with the General outlook most people have on technology. In a 1996 \textit{Washington Post/Kaiser Family Foundation/Harvard} poll, 70\% of Americans said the increased use of technology in the workplace was good for the economy. In a 2010 \textit{Allstate/National Journal} poll, 79\% said that information technology was extremely or very important to creating economic growth in the U.S. Some would say they aren't wrong but it should be noted that it is rare that general public opinion is aligned with the most ethical option.

Even though AI forms only a small section of this problem, it engenders the primary difference between the \textit{First Industrial Revolution} and now, the fact that mental processes can be performed far more efficiently by computers. Technology Heavyweights like Elon Musk and Stephen Hawking have openly spoken about the perils of the technology when it is used for warfare\cite{aidangers}. With some expressing concern about the possibility of it being the cause of human extinction. The extreme views on this issue are well documented, and we will further elaborate on them in the next section, but our aim remains to balance these views to ensure a smooth transition. There is no way to , the best we can do is ensure it is done in the most ethical fashion, without favoring one stakeholder over the other and ensuring that we have a holistic view of the situation rather than proposing solutions that would be suited to a vacuum.

%%% Local Variables:
%%% mode: latex
%%% TeX-master: "main"
%%% End:
