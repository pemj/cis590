\section{The Current Scenario}
\label{sec:-current}
On the $12^{th}$ of March 2016, history was made. Lee Sedol, a 
professional Go Player who some consider to be the best player of the 
last decade, lost a game of Go. His third consecutive loss in a 
best-of-five competition. His opponent was AlphaGo, an Artificial 
Intelligence designed to play Go\cite{alphagolee}.  To anyone paying 
attention to this field, it does not seem terribly surprising.  After 
all, IBM's Watson managed to outsmart previous human jeopardy 
champions,\cite{watsonjeopardy}, and it's been decades since the last 
time a human was able to compete witha chess-playing AI. But there is 
something that separates Go from these other games. 

Firstly, a game of Go can represent a vast array of possible states, 
approximately $10^{768}$, a number more suitable for counting atoms 
in the observable universe than counting Chess positions.  This rules 
out brute-force techniques for solving the game.  That means that we 
can't just solve Go, we need to teach the computer how to actually play 
Go.  This leads to our second point. This is a significant 
milestone in the trend of using computers to replicate mental 
processes that were earlier considered to be the sole dominion of the 
living beings, particularly humans.

As mentioned earlier, we live in the Information Age, and changes have a 
tendency to propagate through our society fairly quickly. Globalization, 
and the subsequent transfers of technology enabled by the Internet, 
allows for a proliferation of ideas and techniques that would have been 
unheard of even a few decades ago. Automation's strongest argument has 
always been that it allows us to reduce manual labor and improve the 
quality of human lives by allowing humans to direct their labors towards
the kind of work that assembly lines cannot replicate: matters of 
intellect.  Some may argue that this is no longer the case.  
Many proponents of automation have pointed to the migration of people 
from unskilled and semi-skilled labor to more skilled labor which 
rely on mental processes that could not have been automated at the 
time. It is evident that this sacred boundary is moving, and, 
depending on whom you ask, may already be at risk of dissolving 
entirely.

The advent of applications that have improved accessibility to services, 
from planning your week to simply setting an alarm or a reminder to do 
your taxes, all these tasks have been automated already. The automation 
of these activities begs the question of whether this would make humans 
obsolete in performing these activities. The public sentiment does not 
seem to reflect this. In a study carried out by the 
\textit{Pew Research Center}\cite{workforcepew}, although about 
two-thirds of Americans believe that computers and robots will be 
performing most of the work being done by humans in the next 50 
years, nearly 80 \% believe that their current jobs will survive the 
transition..
	
A likely answer to this is skewed figure is probably to do with the 
general outlook most people have on technology. In a 1996 
\textit{Washington Post/Kaiser Family Foundation/Harvard} poll, 70\% of 
Americans said the increased use of technology in the workplace was good 
for the economy. In a 2010 \textit{Allstate/National Journal} poll, 
79\% said that information technology was extremely or very important 
to creating economic growth in the U.S. They may not be entirely wrong, 
but it should be noted that the opninion of the general public is rarely 
aligned with the most ethical option.

Even though AI forms only a small section of this problem, it makes up 
the primary difference between the \textit{First Industrial Revolution} 
and what some are calling the beginning of The Second Machine ag, as 
more and more mental processes can be performed by computers with 
greater speed and correctness.  Technology Heavyweights like Elon Musk 
and Stephen Hawking have openly spoken about the perils of the 
technology when it is used for warfare\cite{aidangers}, expressing 
concern about the possibility of human extinction if left unchecked. 
The extreme views on this issue are well documented, as we will further 
elaborate in the next section, but our aim remains to balance 
these views to ensure a smooth transition.  No situations is without 
its winners and losers, so the best we can do is ensure that the current 
march towards an increasingly automated society is done in the most 
ethical fashion. We wish to move forward without favoring the interests 
of some people over others, and ensuring that we emerge with a holistic 
view of the situation.

%%% Local Variables:
%%% mode: latex
%%% TeX-master: "main"
%%% End:
