\subsection{Identifying Stakeholders}
\label{sec:-identify}
Ethical Analysis of an action, circumstance or rule must always begin 
with Stakeholder Analysis. Stakeholders are those who are affected as 
consequence of the action, circumstance or rule. In other words, the 
parties concerned have a stake in the eventual outcome of performing 
the action or implementing the rule.  In our framework, we consider 
a rough categorization of factors that play into our stakeholder 
identification.  Then, based on historical precedent, we observe the 
existence of larger classes of stakeholders. The interplay between 
these classes, in practice, follows historical trends that allow us 
to reason about larger societal effects without getting too bogged 
down in hypotheticals.  That is, wherever we can, we wish to base our 
analysis not just on sound logic, but on historical fact.

The process of identifying Stakeholders is generally specific to the 
action and rule as well as to the circumstances that surround the 
process. Although our framework strives to espouse the Kantian Ideal 
of universality, it would be naive to assume that the stakeholders 
would be a constant across the board over time.  To this end, in order to 
identify the stakeholders, we must try to obtain a 
holistic view of the situation, including our own biases and viewpoints 
in such an analysis. In order to do this, we identify three 
different types of factors that may affect the identification process.
\begin{itemize}
  \item Cultural Bias:	  
    \begin{pointenv} % I don't even have any edits here really, I just dig this section.
      Cultural Bias has long hindered the adoption of technology. Rather than 
      considering this to be a drawback, let us instead see it as nothing more 
      than the decision making process of a collective, based on previous 
      experience and a healthy skepticism about new and sometimes untested 
      methods and technologies. Like a double-edged sword, the same cultural 
      bias might catalyze the adoption of the technology as well. This can be 
      seen in the early problems faced during adoption of disruptor 
      technologies like the radio and aircraft. In order to correctly identify 
      stakeholders, it is imperative to accurately weigh the cultural bias 
      associated with a given society; especially since the effects of the 
      technology might be experienced only by a certain section of that society.
    \end{pointenv}
  \item  Value System:
    \begin{pointenv}
      Value systems play a big part in helping the acceptance of technology. 
      Say for example, in the United States the Apple vs. FBI debate has 
      tested the value systems of the public, where protection of individual 
      liberties is of paramount importance. If, however, a similar case would 
      have taken place in the People's Republic of China, the prevailing value 
      systems would have ensured massive public support for the government's 
      stand. In short, the classification of stakeholders may change 
      dramatically.  In one society, the Government is seen as a representative 
      of the public, where as in another, the individuals making up that public 
      take on an identity of their own.  In one society, people expect 
      a very particular type of privacy from the society they live in, and 
      when that privacy is violated, they experience harm.  In a society 
      where the social contract is different, where people have different
      expectations of privacy, they may not perceive such a harm if they 
      find themselves in the same circumstances.
    \end{pointenv}
  \item Economic Status:
    \begin{pointenv}
      The economic status of a society broadly refers to the distribution 
      of wealth in the society. We see vastly contrasting structures in 
      Developed Countries and in Developing Countries. One example that 
      could elucidate this difference is the adoption of automation of 
      jobs such as customer service.  In developed countries, these are 
      considered tertiary level jobs and do not form the backbone of the 
      economy. Hence the effect would be largely mitigated, and with 
      government intervention, could be avoided nearly entirely. But in 
      the case of developing countries, where semi-skilled jobs like 
      freight services employ a large number of people, we would see a 
      larger effect in terms of unemployment. Accounting for this would 
      play a large part in correctly weighing each stakeholder.
    \end{pointenv}
\end{itemize}

%%% Local Variables:
%%% mode: latex
%%% TeX-master: "../main"
%%% End:
