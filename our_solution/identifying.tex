\subsection{Identifying Stakeholders}
\label{sec:-identify}
Ethical Analysis of an action, circumstance or rule must always begin with Stakeholder Analysis. Stakeholders are those who are affected as consequence of the action, circumstance or rule. In other words, the parties concerned have a stake in the eventual outcome of performing the action or implementing the rule.

The framework aims to take into account two main factors; firstly, factors that affect the identification of stakeholders; and secondly, when viewing automation we observe three general classes of stakeholders, with varying levels of interplay seen between these classes based on the aforementioned factors.

The process of identifying Stakeholders is generally specific to the action and rule as well as to the circumstances that surround the process. Although our framework strives to espouse the Kantian Ideal of universality, it would be naive to assume that the stakeholders would be a constant across the board; taking into account that our framework aims to perform an ethical analysis on the effect of automation
    To this end, in order to identify the stakeholders we must try to obtain a holistic view of the situation. In order to do this, we enumerate three factors, which would have an effect in the identification process:
    \begin{itemize}
      \item Cultural Bias:	  
\begin{pointenv}
Cultural Bias has long hindered the adoption of technology. Rather than attributing this as a drawback (which it is sometimes) see it as nothing but the decision making process of a collective, based on previous experience as well as skepticism about new and sometimes untested methods/technologies. Like a double-edged sword, the same cultural bias might catalyze the adoption of the technology as well. This can be seen in the early problems faced during adoption of disruptor technologies like the Radio and aircrafts. In order to correctly identify stakeholders, it is imperative to accurately weigh the cultural bias associated with a given society; especially since the effects of the technology might be experienced only by a certain section of society.
\end{pointenv}
      \item  Value System:
\begin{pointenv}
Value systems play a big part in helping the acceptance of technology. Say for example, in the United States the Apple vs. FBI debate has tested the value systems of the public, where protection of individual liberties is of paramount importance, if a similar case would have taken place in the People's Republic of China, the prevailing value systems would have ensured massive public support for the Government's stand. In short, the classification of stakeholders would change dramatically where in one society, the Government represents the interests of the public where as in another, the individual/public takes on an identity of their own.
\end{pointenv}
      \item Economic Status:
\begin{pointenv}
The Economic Status of a society broadly refers to the distribution of wealth in the society. We see vastly contrasting structures in Developed Countriies and in Developing Countries. One example that could elucidate this difference is the adoption of automation of jobs such as customer service. This would have varying effects depending on the country. In developed countries, these are considered tertiary level jobs and do not form the backbone of the economy. Hence the effect would be largely mitigated and with government intervention could also be avoided. But in the case of developing countries, where semi-skilled jobs like freight services employ a large number of people, we would see a larger effect in terms of unemployment. Accounting for this would play a large part in correctly weighing each stakeholder.
\end{pointenv}
    \end{itemize}

%%% Local Variables:
%%% mode: latex
%%% TeX-master: "../main"
%%% End:
