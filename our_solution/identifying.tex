\subsection{Identifying Stakeholders}
\label{sec:-identify}
Ethical Analysis of an action, circumstance or rule must always begin with Stakeholder Analysis. Stakeholders are those who are affected as consequence of the action, circumstance or rule. In other words, the parties concerned have a stake in the eventual outcome of performing the action or implementing the rule.\\
The framework aims to take into account two main factors, first, factors that affect the identification of stakeholders; and secondly, when viewing automation we observe three general classes of stakeholders, with varying levels of interplay seen between these classes based on the aforementioned factors.
\begin{enumerate}[a]
  \item \textbf{Factors affecting Stakeholder Identification}:

    The process of identifying Stakeholders is generally specific to the action, rule as well as the circumstances that surround the process. Although our framework strives to espouse the Kantian Ideal of Universality, it would be naive to assume that the stakeholders would be the constant across the board. Taking into account that our framework aims to perform an ethical analysis on the effect of automation
    To this end, in order to identify the stakeholders we must try to obtain a holistic view of the situation, in order to do this we enumerate three factors, which would have an effect in the identification process:\\
    \begin{itemize}
      \item Cultural Bias:\\
        Cultural Bias has long hindered the adoption of technology. Rather than attributing this as a drawback, which it is sometimes, it is nothing but the decision making process of a collective, based on previous experience as well as skepticism about new and sometimes untested methods/technologies. Like a double-edged sword, the same cultural bias might catalyze the adoption of the technology as well. This can be seen in the early problems faced during adoption of disruptor technologies like the Radio and aircrafts. In order to correctly identify stakeholders, it is imperative to correctly weigh the cultural bias associated with a given society especially since the effects of the technology might be experienced only by a certain section of society.
      \item  Value System:\\
        Value systems play a big part in helping the acceptance of technology. Say for example, in the United States the Apple vs. FBI debate has tested the value systems of the public, where protection of individual liberties is of paramount importance, if a similar case would have taken place in the People's Republic of China, the prevailing value systems would have ensured massive public support for the Government's stand.\\
        In short, the classification of stakeholders would change dramatically where in one society the Government represents the interests of the public where as in another the individual/public takes on an identity of their own.
      \item Economic Status:\\
        The Economic Status of a society broadly refers to the distribution of wealth in the society. We see vastly contrasting structures in Developed Countriies and in Developing Countries. One example that could elucidate this difference is the adoption of automation of jobs like customer service would have varying effects depending on the country. In developed countries, these are considered tertiary level jobs and do not form the backbone of the economy hence the effect would be largely mitigated and with government intervention could also be avoided. But in the case of developing countries, where semi-skilled jobs like freight service employ a large number of people, we would see a larger effect in terms of unemployment. Accounting for this, would play a large part in correctly weighing each stakeholder.
    \end{itemize}
  \item \textbf{General Classes of Stakeholders}\\
    Taking into account the effect of automation over the entire cross-section of society, we think it is prudent to identify certain classes of stakeholders which would be present across any situation, regardless of the geographical location, economic status and value system. Although there would different interplays between these classes given the above mentioned factors that affect the selection of stakeholders, their existence is no less in doubt by our standard.\\
    It is important to add that although we define these classes, they are in no way exhaustive and the due diligence required to identify stakeholders must be done in order to ensure the holistic view we would like to achieve in this part of the process.
    \begin{itemize}
      \item \textbf{Public:}\\
        The Public, refers to society at large who no matter what the situation will be directly/indirectly affected by the implementation of the automation. Typical issues include job/labor concerns, concerns regarding misuse against other members of the public and concerns regarding laws enforcing the ethical use of the same, just to name a few. By far this is one of the larger classes, and within it many more that might emerge, especially when the technology is targeted to a certain section/strata of society. It is therefore essential to identify which section might have a larger stake in the implementation of the automation, be it positive or as it is often seen, negative.
      \item \textbf{Government:}\\
        Government, refers to the regulatory body that oversees the ethical use of the automation. We can say in certain scenarios, the lines discerning public interest and government interest might blur, we considered these to be exceptional cases where we can identify one class to represent the interests and stakes of another(as mentioned in the example above). It can be seen in most modern societies today, the public does not necessarily gain adequate representation of their interests through government and similarly, the Government must at times ensure the security and well-being of the state even if that does not represent what the public deems fit. For this and reasons that outline similar cases of dichotomy, we would like to represent this as a separate class.
      \item \textbf{Corporate Interests:}\\
        Automation today, as it was earlier as well, is driven by commercial interests. We represent these interests through the class of corporate stakeholders. These would represent the individuals as well as organizations which have driven the case for the automation and would benefit in some monetary fashion by the said implementation. Apart from this, we would like classify industries in this class of stakeholders as well, because of the widespread transmission of ideas and technology, practices linked with an increase in productivity would be widely adopted even if that is at the cost of jobs. Although it would be easy to classify corporate interests as strictly monetary, but in certain scenarios we see that this stakeholder has aligned interests with another class(such as Government) and hence would require a somewhat different approach to weighing their stakes.\\ 
    \end{itemize}
\end{enumerate}


%%% Local Variables:
%%% mode: latex
%%% TeX-master: "main"
%%% End:
