\subsection{General Classes of Stakeholders}
\label{sec:-general}
Taking into account the effect of automation over the entire cross-section of society, we think it is prudent to identify certain classes of stakeholders which would be present across any situation, regardless of the geographical location, economic status and value system.  We do not posit that these general classes stand alone, but rather that an exmaination of the interplay between the interests of these classes is an important first step for any stakeholder analysis we wish to undertake.
It is important to add that these divisions are our own taxonomy, embraced because it is useful for our purposes, not because it is necessarily representative at all levels.We emphasize that the due diligence required to identify stakeholders must be done in order to ensure the holistic view we would like to achieve in this part of the process.
\begin{itemize}
\item Public:
\begin{pointenv}
The Public, refers to society at large who no matter what the situation will be directly/indirectly affected by the implementation of the automation. Typical issues include job/labor concerns, concerns regarding misuse against other members of the public, and concerns regarding laws enforcing the ethical use of the same, just to name a few. By far this is one of the larger classes, and within it many more might emerge, especially when the technology is targeted to a certain section/strata of society. It is therefore essential to identify which section might have a larger stake in the implementation of the automation, be it positive or as it is often seen, negative.
\end{pointenv}
\item Government:
\begin{pointenv}
Government, refers to the regulatory body that oversees the ethical use of the automation. We can say in certain scenarios, the lines discerning public interest and government interest might blur, we considered these to be exceptional cases where we can identify one class to represent the interests and stakes of another(as mentioned in the example above). It can be seen in most modern societies today, the public does not necessarily gain adequate representation of their interests through government and similarly, the Government must at times ensure the security and well-being of the state even if that does not represent what the public deems fit. For this and reasons that outline similar cases of dichotomy, we would like to represent this as a separate class.
\end{pointenv}
\item Corporate Interests:
\begin{pointenv}
Automation today, as it was earlier as well, is driven by commercial interests. We represent these interests through the class of corporate stakeholders. These would represent the individuals as well as organizations which have driven the case for the automation and would benefit in some monetary fashion by the said implementation. Apart from this, we would like classify industries in this class of stakeholders as well, because of the widespread transmission of ideas and technology, practices linked with an increase in productivity would be widely adopted even if that is at the cost of jobs. Although it would be easy to classify corporate interests as strictly monetary, but in certain scenarios we see that this stakeholder has aligned interests with another classes(such as Government) and hence would require a somewhat different approach to weighing their stakes.
\end{pointenv}
\end{itemize}
%%% Local Variables:
%%% mode: latex
%%% TeX-master: "main"
%%% End:
