\subsection{Defining a Coherent Value System}
\label{sec:-value}
Subsequent to identifying stakeholders and classifying the stakeholders into the basic classes, or new ones depending on the circumstances, we must elucidate upon a value system that would act as a guide and in the process allow us to reach a conclusion regarding the ethical implementation of the automation in question. \\
We propose a negative utilitarian framework focused on preventing harm, where we define harm in terms of violation of a number of values that we consider to be important.  We apply this framework with the caveat that its usefulness may break down at certain moral extrema, which result in a reluctance to make trade-offs between particular harms in particular consequences.
We identify a rough taxonomy of harms as violations of the following values for stakeholders in an ethical analysis:
\begin{enumerate}
  \item Autonomy of Self
    \begin{pointenv}
      Autonomy of self, refers to the basic human rights as laid down by the Universal Declaration Human Rights\cite{assembly1948universal}, which includes but is not limited to, Right to life, Right to freedom of expression\\
      As laid-down in the Charter, any harm to these basic rights can considered a violation of our framework, and the person responsible for the analysis should eschew any automation that endangers these basic rights, or in the worst case to mitigate these violations through implementation of other practices that would ensure protection of these rights to the highest degree possible.\\
      A vital point within this tenet is the Right to Privacy, although most people believe this is a fundamental right, in a majority of  cases this would take precedence over any technology that infringes on this right. As we are seeing the battle rage between Apple and the FBI, the protection of this entire section of rights would come under the social contract between the individual/s and society at large. There is requirement to state the importance of the protection of these rights, because sometimes these rights may not be protected by the social contract mentioned earlier. 
    \end{pointenv}

  \item Welfare/Well-Being
    \begin{pointenv}
      The ability to ensure Welfare and well being of self, family and society at large has long been considered the highest responsibility. The world we live in today, this is directly associated with sustenance as well as health.\\
      Any violation of the welfare of the stakeholders through either contamination of sources of sustenance or loss of livelihood could be considered a violation of this principle. Understandably, there are costs to progress, although this framework aims to minimize the infringements, we propose that on finding a violation, action should be taken against the party to ensure that the cost of implementing the automation would be higher than the gain from such technologies, which are implemented by violating these rules.\\
      The purpose of this framework is less to take a moral stand but rather a way to balance the effect of consequences on the stakeholders, by suggesting action we believe that the framework used in conjunction with the ground reality would help the user come to an understanding of how to ensure a balance is reached. 
    \end{pointenv}

  \item Ecoomic Sufficiency
    \begin{pointenv}
      The above points form moral values we wish to preseve in the 
      course of any decisions made within the context of our moral framework.  
      In the course of so doing, we believe it is important to note that 
      many harms or benefits to the well-being and autonomy of individuals 
      we must consider are economic in nature.  That is, they may be 
      considered in terms of costs leveraged upon their persons, taxes 
      leveled against corporations, tax incentives provided to 
      farmers that grow particular crops, and other similar structural 
      changes.  Therefore, when attempting to analyze harm, it would 
      behoove us to consider the economic outcome, and how it may inform 
      the other values we wish to preserve in our efforts to make 
      ethical decisions.
    \end{pointenv}
\end{enumerate}

%%% Local Variables:
%%% mode: latex
%%% TeX-master: "main"
%%% End:
