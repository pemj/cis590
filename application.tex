\section{Self-Driving Cars: an Application of our Framework}
\label{sec:-application}

\subsection{Background}
Long a fascination for inventors, the seed for autonomous cars was sewn 
in the 1960s with the Stanford Cart\cite{moravec1990stanford}. The cart 
navigated small spaces by taking photos of its surroundings and using a 
simple computer program to analyze the photos and choose a path of 
travel. Two decades later, a vision-guided Mercedes-Benz van designed 
by Ernst Dickmanns navigated empty streets at speeds of up to 39mph. By 
2009, Google began privately testing anonymous vehicles on closed 
streets, and by 2012, changes in various states' laws allowed for 
testing on city streets. As of 2016, Google claims over 1.3 million 
autonomous miles have been driven, although Google's cars are limited 
to 25mph and must always have a human driver present.\cite{wired}  As 
driverless car technology improves and becomes more widely implemented, 
the true societal impact of this technology will be felt.


\subsection{Application}

We're in an interesting time in regard to autonomous vehicle technology. 
It's clear that the technology will become widely used, but it has yet 
to actually happen. This presents an opportunity to consider how, as a 
society, we should approach the ethical ramifications of this 
potentially society-altering technology. To do so, we'll apply the 
ethical framework laid forth earlier in this paper to autonomous cars.

{\large three main stakeholders: public, government, corporations}

To begin, it is important to identify the three main stakeholders in 
driverless car technology: the public, the government, and corporations. 
The stakeholders are universal, but each stakeholder's value systems and 
the weight given to each stakeholder can vary based on cultural norms 
and/or economic status. In the interest of concision, this application 
will focus on a generalized idea of American culture, values, and 
economic status.

Because the United States is such a demographically, economically, and 
culturally diverse country, it is difficult to make generalizations of 
the country as a whole. That being said, it is possible to extrapolate 
mainstream value systems of its public, government, and corporations. 
The public values individualism, corporations value growth and the 
acquisition of capital/power, and the government acts as a mediator. 
Each stakeholder has the power to act on the other and force change. 
With the stakeholders, value systems, and power structure identified, 
it is possible to apply our ethical framework to autonomous vehicle 
technology in the United States.

{\large Do self-driving cars satisfy a utilitarian?}

Autonomous vehicles have a lot to offer a utilitarian. A recent 
Virginia Tech study\cite{blanco2016automated} shows that the crash 
rate of Google's self-driving cars is lower than the national crash rate 
of conventional cars. Google's cars experience 3.2 crashes per million 
miles while conventional cars experience 4.2 crashes per million miles. 
With autonomous vehicle technology continually improving, the technology 
will become even safer in the years to come. But is a full embrace of 
the technology the best course of action? For a utilitarian, this may 
be an easy answer, and therein lies a problem: autonomous vehicles face 
a variety of moral dilemmas, and a one-size-fits all ethical framework 
doesn't cover all of these possibilities.

{\large Example 1:}

Consider the following example: during the November 2015 terror attacks 
in Paris, ride-sharing app Uber's algorithmic design dramatically raised 
prices in the areas near the attacks due to an increase in ride requests 
from people trying to flee the chaos. In this situation, the public and 
the corporation were in direct opposition to each other. Uber saw the 
situation as a moneymaking opportunity, while the public saw Uber as 
price gouging and profiting from a disaster. Eventually, Uber gave in 
to public demand and lowered prices. Situations such as this will become 
the norm in the age of autonomous cars. In fact, Uber is aggressively 
seeking driverless car technology for use in their taxi company.

{\large How would our framework handle this scenario?}

The moral framework presented in this paper allows for exceptions that 
strict utilitarianism does not. In the case of Uber raising the prices 
on people fleeing a disaster, utilitarianism supports profit 
maximization as a means to the greatest societal good. Conversely, our 
framework allows an exception on the basis of protecting an 
individual's welfare and well-being.

{\large Example 2:}

Consider a hypothetical example: you're riding down the highway in your 
autonomous car and a large object falls off of a truck in front of you. 
Your car cannot stop itself in time and must decide on the best course 
of action. If it continues straight and runs into the object, it places 
your life in serious jeopardy. There are two motorcyclists on either 
side of your car; one is wearing a helmet and the other is not. Swerving 
into one of the motorcyclists will likely save your life, but jeopardize 
the life of one of the motorcyclists.

With strict utilitarianism, your car would likely swerve into the 
motorcyclist wearing the helmet because it creates the best chance of 
survival amongst all of the possible scenarios, but is it fair to punish 
a motorcyclist for being responsible and wearing a helmet? What are the 
legal ramifications of such a programmed decision? In such a scenario, 
our ethical framework would fall back onto the autonomy of the individual 
to decide how their car is programmed to react in cases such as these.

Conclude that using our framework makes self-driving cars more morally 
acceptable.

While far from foolproof, our ethical framework allows for important 
exceptions to utilitarianism in extreme cases. The implementation of 
driverless cars will undoubtedly lead to many such extreme cases, 
therefore the existence of “fallback” rules will be critical. In the 
previous examples, the “fallback rules” of our ethical framework placed 
health and well-being above financial profit as well as the autonomy of 
a human over the autonomy of a machine. These are only two examples of 
the types of moral dilemmas that will become ever-present in our 
society, and it is critical that our society agree on an even more 
extensive ethical framework to confront these issues.


%%% Local Variables:
%%% mode: latex
%%% TeX-master: "main"
%%% End:
