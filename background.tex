\section{Background}
\label{sec:-background}

Later on in our paper, we will put forth an ethical framework for use 
in answering questions pertaining to automation and related technologies. 
As an important part of that framework, we aim to analyze who may be 
affected by changes in the state of the art of technology, and how 
those effects may ripple out to change society at large.  While these 
analyses would be much simpler were it not the case, the fact of the 
matter is that technological innovation is part and parcel of human 
society.  Advances in technological innovation cannot be extricated from 
the social conditions which gave rise to such innovation, nor can 
they be separated from the social consequences which result from 
the adoption of that technology.  If we wish to base 
our analyses in historical fact, it seems prudent to first review some 
basic history with an eye towards the factors we intend to consider.  
If we wish to see a preview of the impacts that automation will have 
on the workforce, and consequently the society at large, it seems 
prudent to look at the ways automation shaped society as its effects 
on the economy were first made manifest.

Let us consider, for a moment, the steam hammer. The steam hammer, a 
powered hammer for forging and 
stamping of metal parts, was invented in the mid-1800s as increasingly 
large and complex machines necessitated the fabrication of larger and 
heavier components.  It made routine the construction of many works 
that, only years before, would have been considered marvels of 
engineering and concerted human effort.  It contributed to the construction 
of great ships, the production of delicate clockwork, and a massive 
increase in the industrial productivity of Western Europe that saw an 
increase in the availability of cheaply manufactured goods formerly 
restricted to only the richest members of society.  It also 
contributed\cite{gaskell1833manufacturing} to a sharp decrease in the 
leverage that skilled engineers held over their employers, resulting in 
drastic consequences for working environments, and sharp increases in 
child-labor.  James Nasmyth, one of those credited with the invention 
of the steam hammer, made the following statement regarding the power 
of automation in the modern (as of 1851) factory:

\begin{quotation}\small
  ``The characteristic feature of our modern mechanical 
  improvements is the introduction of self-acting machinery. What every 
  mechanical workman has now to do, and what every boy can do, is not to 
  work himself, but to superintend the beautiful labour of the machine. 
  The whole class of workmen that depend exclusively upon their skill is 
  now done away with. Formerly, I employed four boys to every machine. 
  Thanks to these new mechanical combinations, I have reduced the number 
  of grown-up men from 1,500 to 759. The result was a considerable 
  increase in my profits.''\cite{marx1867strife}
\end{quotation}

From the specific, we proceed to the general.  The example of the steam 
hammer (in particular, the contrast between its positive and negative 
social effects) raises one of the first difficulties we will be forced to 
surmount in the course of developing our ethical framework.  
In the case of the industrialization of the workforce of Western 
Europe, those who wished to deploy greater 
degrees of automation in their factories were directly at odds 
with those who work in those factories.  This is reflective of 
the effects of automation, not just on the market for the product they 
create, but also on the labor market.  Labor, like any other commodity, 
has a price, which fluctuates based on supply and demand.  When the 
owners of the factory can produce just as much value from less work, 
the demand for that labor goes down, and its price drops.  Those 
workers have less leverage, and, as observed by Gaskell, their working 
conditions suffer.

When balancing the gains in technology and standards of living that 
have resulted from this march of progress, we see that there are two 
sides of this coin we must consider.  The society as a whole may 
benefit, but what of the workers that toil in the factory?  Does the 
gain outweigh the harm?  Fortunately, in this particular case, we are not 
forced to make such a decision: social and cultural pressures intervened.  
The labor movement, in an internationally 
coordinated effort, slowly but surely pushed many governments 
into enacting legislation that protected workers from unfair practices 
on the part of their employers.

And yet, we are faced with a somewhat tougher questions, when we step 
back and consider the broader context around the industrialization of 
Western Europe.  Industrialization is a somewhat titanic effort, it 
requires the investment of massive amount of capital to begin to see 
the advantages resulting from leveraging the economies of scale that 
make automation profitable.  Different societies have achieved such 
a concentration of wealth in different ways.  China, in the course of 
The Great Leap Forward, began industrialization on a massive scale by 
nationalizing all production, taking central command of labor and 
distributing it according to a grand plan for industrialization of the 
workforce as a whole.  The Soviet Union took a similar approach, 
nationalizing industry and placing it under central command, but then 
decentralized the industry geographically, exporting raw materials to 
nearby states as it absorbed them, thus creating a demand for the 
growth of industry in those formerly independent states that drove 
their growth further outward until they started bumping up against 
NATO countries.  Western European countries accumulated wealth in 
privately held trading companies by establishing an early foothold 
in mercantile capitalism through the Triangle Trade. That is, they 
establishing trade routes between Europe, Africa, and the Americas, 
alternating in the shipment of slaves, raw materials (such as sugar 
cane grown by slaves), and manufactured goods (such as rum produced 
from last year's sugar cane).  

In each of these situations, there exists a contradiction 
between the forces acquiring enough wealth to assemble an industrial 
base, and the people that had to give up that wealth.  In China and 
the Soviet Union, the costs were paid by the citizens of those 
countries, and they reaped the benefits slowly as their standards of 
living increased fairly evenly.  In Europe, the standards of living 
and availability of luxury goods skyrocketed much more rapidly: after 
all, the cost of automaton was offloaded to somewhere else: Africa and 
the American colonies.  In the Soviet Union and China, the analysis looks 
simple: did the country gain more than it lost?  China suffered a number 
of famines as the Great Leap Forward encountered organizational 
difficulties, but over a period of 30 years, the average lifespan of a 
Chinese citizen nearly doubled.  

When, however, we look at a situation like British industrialization, 
we now find that we must take a look at a much more diverse selection 
of groups in analyzing the impact of technology on the society.  As 
globalization has blurred the lines between nations, so too has 
information technology led to a vastly more interconnected world than 
we have ever seen.  A change in technology today can lead to nearly 
instantaneous changes through enormous swaths of society.  In the 
immediately following sections, we will further examine the current 
state of the world as it has evolved, the approaches that some 
groups take in analyzing that world, and the steps they deem necessary 
due to such an analysis.

%%% Local Variables:
%%% mode: latex
%%% TeX-master: "main"
%%% End:
