\section{Background}
\label{sec:-background}

Later on in our paper, we will put forth an ethical framework for use 
in answering questions related to automation and related tecchnologies. 
As an important part of that framework, we aim to analyze who may be 
affected by changes in the state of the art of technology, and how 
those effects may ripple out to change society at large.  While such 
an analysis would be much simpler were it not the case, the fact of the 
matter is that technological innovation is part and parcel of human 
society, and advances in such technology cannot be extricated from 
the social conditions that led to those advances, or the consequences
that result from those advances.  Indeed, this very fact is the 
motivation behind this paper in the first place.  If we wish to base 
our analyses in historical fact, it seems prudent to first review some 
basic history with an eye towards the factors we intend to consider.  
If we wish to see a preview of the impact that automation will have 
on the workforce, and consequently the society at large, it seems 
prudent to look at the ways automation shaped society as its effects 
on the economy were first made manifest.

It seems prudent to start with a look at a single, practical example 
of the early industrial process, proceeding thereafter to the broader 
patterns engendered by such an invention: To that end, let us consider 
the steam hammer.  The steam hammer, a powered hammer for forging and 
stamping of metal parts, was invented in the mid-1800s as increasingly 
large and complex machines necessitated the fabrication of larger and 
heavier components.  It made routine the construction of many works 
that would have been considered marvels of engineering and concerted 
human effort, only years before.  It contributed to the construction 
of great ships, the production of delicate clockwork, and a massive 
increase in the industrial productivity of Western Europe.  It also 
contributed\cite{gaskell1833manufacturing} to a sharp decrease in the 
leverage that skilled engineers held over their employers, resulting in 
drastic consequences for working environments, and sharp increases in 
child-labor.  James Nasmyth, one of those credited with the invention 
of the steam hammer, made the following statement in regard to the 
machinery he had helped introduce to the world in wake of widespread 
strikes in the 1850s:

\textquote{The characterstic feature of our modern mechanical 
improvements is the introduction of self-acting machinery. What every 
mechanical workman has now to do, and what every boy can do, is not to 
work himself, but to superintend the beautiful labour of the machine. 
The whole class of workmen that depend exclusively upon their skill is 
now done away with. Formerly, I employed four boys to every machine. 
Thanks to these new mechanical combinations, I have reduced the number 
of grown-up men from 1,500 to 759. The result was a considerable 
increase in my profits.}\cite{marx1867strife}

From this example, we may consider the fact that 

 

{\large OUTLINE}


Talk about how social issues are intimately connected to automation 
and industry.  That is, we cannot disconnect any innovation from the 
changes it drives in society, and we seek to give examples of that sort 
of thing. (check)

compare industrial revolution to automation and such (check)

Talk about the Steam Hammer.

Talk about how economics, and the concept of the economy of scale, 
demands ever-increasing levels of efficicency and innovation

(probably connect that to primitive accumulation to connect to the next bit)

Related to that, talk about the need for a concentration of capital.  
Specifically, discuss how Japan, the Soviet Union, China, and the 
West achieved it differently, and the costs related to each of those 
situations.

Talk about the legacy of those approaches, and how they inform our 
current economic status and sociopolitical climate.  Name the winners 
and losers, and mention that we're going to keep an eye on history when 
we get to the point of identifying stakeholders.  

Probably mention 
Elizaeth's point, in that there are many current groups who see the 
path we're on as reflective of certain different parts of history.  
That is, they each take particular lessons from that history, and 
the lesson they consider to be most important tells us something 
about their approach.




%%% Local Variables:
%%% mode: latex
%%% TeX-master: "main"
%%% End:
