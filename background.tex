\section{Background}
\label{sec:-background}
{\large OUTLINE}


Talk about how social issues are intimately connected to automation 
and industry.  That is, we cannot disconnect any innovation from the 
changes it drives in society, and we seek to give examples of that sort 
of thing.

compare industrial revolution to automation and such

Talk about how economics, and the concept of the economy of scale, 
demands ever-increasing levels of efficicency and innovation

(probably connect that to primitive accumulation to connect to the next bit)

Related to that, talk about the need for a concentration of capital.  
Specifically, discuss how Japan, the Soviet Union, China, and the 
West achieved it differently, and the costs related to each of those 
situations.

Talk about the legacy of those approaches, and how they inform our 
current economic status and sociopolitical climate.  Name the winners 
and losers, and mention that we're going to keep an eye on history when 
we get to the point of identifying stakeholders.  

Probably mention 
Elizaeth's point, in that there are many current groups who see the 
path we're on as reflective of certain different parts of history.  
That is, they each take particular lessons from that history, and 
the lesson they consider to be most important tells us something 
about their approach.




%%% Local Variables:
%%% mode: latex
%%% TeX-master: "main"
%%% End:
