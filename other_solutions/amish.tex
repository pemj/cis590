\subsection{The Amish}
  The Amish in most popular depictions are painted as rural farmers. They are often protrayed as a culture who fear technology. However, this is a gross simplification of the relationship the Amish have with technology. Previously we have seen Bill Joy who is worried about the juxtaposition of human nature, underestimating the probability of catestrophic effects, and technology and Jason Lanier who critiques what he sees as the rampant corporatism with technology. In contrast, the Amish are not so worried about negative effects like pollution, but the effect of new technologies on their culture. The Amish view of technology is that technology and society or culture are created together and in the words of Sheila Jasanoff "both embeds and is embedded in social practices,identities, norms, conventions, discourses, instruments, and institutions " \cite{wetmore2007amish}. The Amish regard technology as a force that has the power to reinforce social norms, control how people interact, or enable new forms of interaction between people. Where Bill Joy implores us to accurately weigh the net harm including the likelihood of catestrophic events with respect to technology and Jason Lanier insists that we need to insure that corporations do not gain most of the benifit of new technologies instead the Amish insist that any new technology is in line with their culture as well as reinforces their tradition, religion, and culture regardless of any objective potential benefits or harm associated with that technology. 
\label{sec:-amish}

%%% Local Variables:
%%% mode: latex
%%% TeX-master: "main"
%%% End:wa
