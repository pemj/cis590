\subsection{The Amish}
\label{sec:-amish}
he Amish are well-known as rural farmers and are notorious for their portrayal as a culture who fears technology. This is nothing short of an oversimplification of the relationship the Amish share with technology. In contrast to the positions taken by Bill Joy and Jason Lanier regarding technology, the Amish are more concerned with the effect of new technologies on their culture than just by it's negative effects on society. The Amish view of technology is that technology and society or culture are created together and in the words of Sheila Jasanoff "both embeds and is embedded in social practices,identities, norms, conventions, discourses, instruments, and institutions " \cite{wetmore2007amish}. The Amish regard technology as a force that has the power to reinforce social norms, control how people interact, or enable new forms of interaction between people. Where Bill Joy implores us to accurately weigh the net harm including the likelihood of catastrophic events with respect to technology and, Jason Lanier insists that we need to ensure that corporations do not gain most of the benefit of new technologies; the Amish believe that any new technology is in line with their culture as well as reinforces their tradition, religion, and culture regardless of any objective potential benefits or harm associated with that technology. 
%%% Local Variables:
%%% mode: latex
%%% TeX-master: "main"
%%% End:wa
