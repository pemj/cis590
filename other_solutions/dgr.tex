\subsection{Deep Green Resistance}
\label{sec:-dgr}
Deep Green Resistance is a group primarily identified by their beliefs 
and behavior with regard to specifically environmental matters, rather 
than more general opinions about technology.  Specifically, DGR subscribes 
to the notion that industrial civilization poses too great of a threat 
to life on the planet earth, both human and otherwise, to be allowed to 
survive.  The ultimate goal of DGR is the destruction of industrial 
civilization and a return to an earlier stage of societal and economic 
development.  

DGR's philosophy draws from the Deep Ecology movement, 
which holds that an anthropocentric analysis of ecology, defining other 
forms of life in terms of their utility to humankind, does not do an 
adequate job of describing the complexity of ecological systems.  DGR 
takes this a step further, in declaring all life to be equal to human 
life.  Starting at the problem of inequality among life, they turn to 
ideals similar to the political stance of anarcho-primitivism: the idea 
that human society, industrial capitalism in particular, cannot be 
reformed into a more beneficial form.  

Such a philosophy stems from 
an agreement with Friedrech Engal's seminal anthropological work on 
early civilizations\cite{engels2010origin}, specifically the assertion that 
early gatherer-hunter societies were unable to produce structural 
inequality without the ability to acquire surplus wealth.  Unlike Engals, 
modern Marxists, and even their fellow anarchists, DGR and 
anarcho-primitivists do not believe that a society can ever reach a level 
of egalitarianism and freedom from oppression that they are comfortable 
with.  Rather, they believe that the only truly equal societies that have 
ever existed were only able to function without the trappings of 
civilization.  Therefore, they are willing to advocate any means to right 
the wrongs they see with the world.  To them, technology is not evil 
per se.  Rather, technology is another product of a system with which 
they cannot make peace, and as such, it must fall by the wayside.

Later in our paper, we advance an ethical framework for the analysis of 
decisions regarding technology and automation.  In so doing, we 
emphasize that we must look to the past when examining possible 
consequences of an action, basing, wherever we can, our reasoning on 
historical fact.  In this case, we would be remiss if we did not qualify 
our description of the views of DGR with the caveat that their views 
ultimately reduce to something chillingly familiar.  A return to 
pre-industrial levels of technology would necessitate either the death 
of billions, or the restriction of new births to a fraction of their 
current levels.  Without modern medical technology, it becomes a 
disturbing fact that disabled people will not be able to survive in 
such a world: this means DGR endorses either genocide or eugenics.

%%% Local Variables:
%%% mode: latex
%%% TeX-master: "../main"
%%% End:
