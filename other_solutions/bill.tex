\subsection{Bill Joy}
  In recent years a new movement has begun to form in our society, one 
  that most closely resembles the historical Luddite movement, but is 
  different in some important respects. This movement is known as 
  tech-skepticism, or neoluddism. Perhaps the most well known voice in 
  this movement is Bill Joy, co-founder of Sun Microsystems. In 2000, 
  he wrote an article that that reverberated through the tech industry, 
  entitled "Why the Future Does Not Need Us"\cite{joy2000future}. In this 
  article, he articulated views held by many others that might have 
  described themselves as tech skeptics. While he admited that there are 
  fantastic benefits to technology, he made the point that there are 
  also astronomical risks involved, risks we cannot afford to downplay. 

  In particular, he highlights that new technologies have a risk factor 
  that previous technologies did not: robots, engineered organisms, and 
  nanomachines all have the potential to self-replicate.  Joy views this 
  as an extremely dangerous threat, comparing the advent of genetics, 
  nanotechnology, and robots (GNR) to the to the development of weapons 
  of mass destruction of the previous era. Joy points out that the 
  WMDs of the 20th century were powerful threats, but were ultimately 
  self-limiting.  Nuclear weapons, Joy noted, required access to highly 
  regulated raw materials, and shares requirements with biological and 
  chemical weapon development programs for expensive large-scale labs and 
  facilities.  

  At the heart of Bill Joy's argument is that the real difference between 
  20th and 21st century technologies is that these technologies are 
  within the grasph of individuals or small groups. The new paradigm 
  allows what he terms ``knowledge enabled mass destruction,'' or KMD, 
  which is only amplified by the ability of these technologies to self 
  replicate. Joy speaks of the threat of extinction that these new 
  technologies may bring, estimated by at something exceeding 
  30\%\cite{joy2000future}. Joy's answer to this dilemma between the 
  fantastic benefits of technology and the astronomical dangers of 
  the same technology diverges from the view that the proponents of 
  singularity take. Bill Joy advocates for determining the course of 
  the human species by looking at our collective values, ethics, and 
  morals to determine which areas of knowledge are too dangerous to be 
  allowed to be pursued and limiting the development of technologies 
  that are too dangerous. Essentially, Bill Joy rejects the views of 
  proponents of the singlarity that the existential risks are worth 
  the possible benefits of pursuing knowledge in key areas. Furthermore, 
  he warns that for technologies once we have developed these technologies, 
  their self-replicating nature will make it impossible to get rid of 
  them.  According to the Joy and the neo-luddites, we are dangerously close to 
  opening Pandora's box, and there's no sure answer as to whether it 
  has any hope at the bottom.


\label{sec:-bill}


%%% Local Variables:
%%% mode: latex
%%% TeX-master: "../main"
%%% End:
