\subsection{Bill Joy}
  In recent years a new movement has begun to form in our society one that most closely resembles the luddite movement, but is different in some important respects. This movement is known as tech skepticism or neoluddism. Perhaps, the most well known voice in this movement is Bill Joy who was the co-founder, chief scientist, and executive of Sun Microsystems. In 2000, he wrote an article that appeared wired that reverberated through the tech industry and has been cited more than 1000 times according to google scholar entitled "Why the future does not need us" \cite{joy2000future}. In this article, he articulates the views of many tech skeptics that while there are fantastic benefits to technology there are also astronomical risks involved and that it is important to not downplay those risks. Instead those risks must be addressed. \par In particular, he highlights that new technologies have a risk factor that previous technologies did not that is that robots, engineered organisms, and nanobots can all self-replicate. Thus making uncontrolled self-replication a dangerous threat according to Bill Joy. He compares the development of weapons of mass destruction (WMD) based on nuclear, biological, and chemical (NBC) with 21st century technologies genetics, nanotechnology, and robotics (GNR). Joy points out that the WMD's of the 20th century where powerful and a big threat, but nuclear weapons required access to rare and difficult to get raw materials and well protected information of rr a long time. Furthermore, biological and chemical weapons development programs required large-scale activities. However, GNR technologies are far more powerful then the NBC technologies creating a new class of accidents and abuses. At the heart of Bill Joy's argument is that the real difference between 20th and 21st century technologies is that these technologies are within the grasph of individuals or small groups and do not require large facilities or rare raw materials only knowledge. The new paradigm allows what he terms knowledge enabled mass destruction (KMD) which is only amplified by the ability of these technologies to self replicate. Essentially, according to Joy the main danger is that of immense power in the hands of extreme individuals as well as sparking potential weapon's races. Joy speaks of the threat of extinction that these new technologies may bring with them which many estimate to be as high or higher then 30 percent \cite{joy2000future}. Joy's answer to this dilemma between the fantastic 
benefits of technology and the astronomical dangers of the same technology diverges from the view that the proponents of singularity take. Bill Joy advocates for determining the course of the human species by looking at our collective values, ethics, and morals to determine which areas of knowledge are too dangerous to be allowed to be pursued and limiting the development of technologies that are too dangerous. Essentially, Bill Joy rejects the views of proponents of the singlarity that the existential risks are worth the possible benefits of pursuing knowledge in key areas. Furthermore, he warns that for GNR technologies once we have developed these dangerous applications it will be impossible to put them back in the box, because they can be easily copied and do not need to be mined or refined like NBC technologies did. In essence, it is not the 
technologies that Bill Joy fears, but the extremes of human nature. 
\label{sec:-bill}

%%% Local Variables:
%%% mode: latex
%%% TeX-master: "main"
%%% End:
