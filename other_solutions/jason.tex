\subsection{Jason Lanier}
  While tech skeptics like Bill Joy are concerned with the collision of human nature with the capabilities of technologies like 
genetics, robotics, and nanotechnology another movement of tech skepticism is critiquing technology as it is being used right now. Perhaps, one of the most famous members of this movement is Jason Lanier who wrote the best selling book "Who Owns The Future ? " \cite{lanier2014owns}. Jason lanier critiques current and future technology not from the view point of existential risks, but from the view point of how corporations are using technology to usurp most of the benefits. He questions how economic value is determined in the information age. He argues that it is information not manual labor that is the most valuable in today's economy and yet it is precisely this information that we are expected to give away for free. He refers to information here as a concious individual,artistic, or pragmatic contribution to the production of goods, services, cultural output, and he also includes the data that we generate by simply existing and participating in activities online or offline. This information is precisely the information that companies are lahying claim to for free by  providing supposedly free services. For example, companies  like Facebook and Google  whose entire profit structure is based on using the information that they gain from the individuals that make up their customer base. He sees this system as providing disporportionate economic power to the companies that are able to access the most information on people the fastest including their interests, demographic predilections, buying habits, and cyber-movements all in exchange for only "free" admission into social media networks which are becoming so popular that opting out can be detrimental to your career. Instead of the customer people are becoming or already are the product in these systems.  His solution is that people should be paid for the information that they contribute to the mass amounts of data that companies use to profit from social media. Essentially, Jason Lanier critiques technology from the stand point of corporatism instead of human nature. 
\label{sec:-jason}

%%% Local Variables:
%%% mode: latex
%%% TeX-master: "main"
%%% End:
