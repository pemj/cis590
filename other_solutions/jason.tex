\subsection{Jason Lanier's approach}
  While tech skeptics like Bill Joy are concerned with the 
  collision between human nature and the capabilities of technologies 
  like genetics, robotics, and nanotechnology, another movement of 
  tech skepticism is critiquing technology as it is being used right 
  now. One of the more prominent adherents of this movement is Jason 
  Lanier, author of the best selling book ``Who Owns The Future?''
  \cite{lanier2014owns}. Jason lanier critiques current and future 
  technology, not as bearers of existential risks, but as tools that 
  allow their owners to usurp most of technology's benefits. 
  
  He questions how economic value is determined in the information age, 
  arguing that it is information, not manual labor, that is the most 
  valuable in today's economy.  And yet, he opines, it is precisely this 
  information that we are expected to give away for free. For example, 
  companies like Facebook and Google, whose entire profit structure is 
  based on using the information that they gain from the individuals 
  that make up their customer base. He sees this system as providing 
  disproportionate economic power to the companies that are able to 
  access the most information on people the fastest.  These companies 
  take interests, demographic predilections, buying habits, and 
  cyber-movements all in exchange for only "free" admission into the 
  social media networks that have become so popular that opting out 
  can be detrimental to one's career. Instead of the customer, people 
  are the product.  He believes that people should be paid for the 
  information that they contribute, that the vast profits companies 
  make off of our combine information should be shared with the users 
  that make it possible.  To Jason Lanier, technology is not evil, and 
  humans aren't evil.  We have just found ourselves in a system that 
  does not serve our best interests, and Lanier argues that we need 
  to enact comprehensive reform in order to do it.

\label{sec:-jason}
%%% Local Variables:
%%% mode: latex
%%% TeX-master: "../main"
%%% End:
