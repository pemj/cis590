\subsection{The Hope of the Singularity}
\label{sec:-singularity}
 The notion of the singularity was first popularized by computer 
scientist Ray Kurzweil. The singularity is a future time period when 
technological change will occur so rapidly and it's impact will be so 
large that human life will be irrevocably changed to such a degree that
we will be unable to make any predictions about the world beyond the 
singularity. Whether the singularity is dystopian or utopian depends 
really on an individuals point of view. 

The underlying insight behind the singularity is that the rate of 
change of techonology continues to accelerate. Furthermore, the myriad 
ways of leveraging technology are expanding at an exponential rate. The 
singularity is when information-based technologies will include all 
human knowledge and proficiency, including pattern-recognition powers, 
problem solving abilities, and emotional and moral intelligence. 

Ray Kurzweil describes the singularity as that which 
\begin{quotation}\small 
  "represents the culmination of the merger of our biologicacl thinking 
  and existence with our technology, resulting in a world that is still 
  human, but transcends our biological roots. There will be no distinction 
  between human and machine or between physical and virtual 
  reality."\cite{kurzweil2005singularity}. 
\end{quotation}

The claim of singulatarians is that we are in the beginning of the 
transition into the singularity at the knee of the curve. Perhaps the 
most alluring aspect of the hope of Singularity to many is the potential 
to conquer death itself.  They hope to reverse the process of aging using 
biotechnology, and eventually free ourselves from all forms of death 
by uploading our minds into durable machine environments. Singulatarians 
recognize the risks inherent in future technologies such as genetics and 
nanotechnologies, but their approach to new technologies such as AI 
tends to emphasize the benefits over the risks. In their view, scientific 
and technological progress is seen as not just inevitable, but desirable no 
matter the possible cost. 

To a singularitarian, they view their cause as 
nothing less than the most noble pursuit in history: the end of 
all human suffering.  As such, simple problems like poverty or hunger 
are merely edge cases of our existing march towards their goal.  So 
long as it allows us to bootstrap ourselves into the singularity 
faster, what care have we for those that fall along the path, when 
weighed against the whole of the future stretching out before us?  
The singularitarians represent a particular extreme with regard to 
their attitude towards technology.  They acknowledge that technology 
may cause social problems, but claim that future technology will obviate 
those problems and all others.  
  

  

%%% Local Variables:
%%% mode: latex
%%% TeX-master: "../main"
%%% End:
