\subsection{The Hope of the Singularity}
\label{sec:-singularity}
 The notion of the singularity was first popularized by computer scientist Ray Kurzweil. The singularity is a furture 
time period when technological change will occur so rapidly and it's impact will be so large that human life will be 
irrevocably changed. Whether the singularity is dystopian or utopian depends really on an individuals point of view. 
The underlying insight behind the singularity is that the rate of change of techonology continues to accelarate and with it
the power inherent in technology. Furthermore, the abilities of technology are expanding at an exponential rate. The singularity is when information-based technologies will include all human knowledge  and proficiency including pattern-recognition powers, problem solving abilities, and emotional and moral intelligence. According to Ray Kurzweil the singularity will "represents the culmination of the merger of our biologicacl thinking and existence with our technology, resulting in a world that is still human, but transcends our biological roots. There will be no distinction between human and machine or between physical and virtual reality" \cite{kurzweil2005singularity}. By humanity merging with the machines we are in the view of singulartarians reaching our full potential and transcending the limits that our biology that limits us. The claim of singulatarian's is that we are in the beginning of the transition into the singularity at the knee of the curve. Perhaps the most alluring aspect of the hope of Singularity to many is the potential to conquer death itself through reversing aging and conquering it using biotechnology and eventually by uploading our minds into machines. Although, singulatarians recognize the risks inherent in future technologies such as genetics and nanotechnologies their approach to new technologies emphasizes the benefits of new technologies including artificial intelligence as being out weighing the risks in every situation. Essentially, scientific and technological progress is seen as not just inevitable, but desirable no matter the possible cost. The singulartarians represent the extreme with respect to valuing technology over the wellfare of individual humans, society, and the human race as a whole. 
  

  



%%% Local Variables:
%%% mode: latex
%%% TeX-master: "main"
%%% End:
