\documentclass{article}
\usepackage{color}
\usepackage{graphicx}
\usepackage{fullpage}
\usepackage{csquotes}
\pagestyle{empty}
\renewcommand{\labelenumii}{\theenumii}
\renewcommand{\theenumii}{\theenumi.\arabic{enumii}.}
\newcommand{\itab}[1]{\hspace{0em}\rlap{#1}}
\newcommand{\tab}[1]{\hspace{.2\textwidth}\rlap{#1}}
\usepackage{listings}
\usepackage{hyperref}
\begin{document}
\bibliographystyle{IEEE}
\begin{center}
Peter McKay, Nisansa De Silva, Elizabeth Fuller, Bradley Green, Gautam Sondur\\
2016-02-17\\
Project proposal\\
\end{center}

\bigskip
{\Large If Robots Don’t Kill People (and Steal our Jobs), Who Does?}

In our paper (and accompanying presentation), we intend to carry out an 
examination of two related ethical quandaries concerning the increasing 
degree\cite{frey2013future} of automation in the modern world.  First, 
we will consider the ethics of existing systems, exploring the case in 
which automated systems make choices apparently
\cite{johnson2008computers} on their own, and what kind of moral 
reasoning we face in such situations.  Secondly, we will ask a somewhat 
more fundamental question: is it right to automate these tasks in the 
first place?  Rather than trying to achieve answers both general and 
correct, we will instead attempt to lay out a framework which may be 
put to use in the pursuit of answerss for specific questions.


To determine an effective way of reasoning about who is responsible for 
the actions of an automated system, we must first ensure that 
``responsibility'' is the correct term to use, and spend some time 
coming to a common definition such that the rest of the analysis will 
remain coherent.  We intend to touch on both deontological and 
consequentialist notions of responsibility, as well as more recent 
theories prompted by advances in cognitive neuroscience.  We will 
examine how people have managed questions of ethics related to other 
complex systems, including weather systems and bureaucracies
\cite{miller2008collective}.  We will also consider a number of thought 
experiments related to this question, drawing from both sociological 
and literary\cite{bostrom2003ethical} sources.


To assemble a framework for discussing of the ethics of the automation 
of labor, we will start with the basics, considering definitions of the 
relationship between workers and automation\cite{marx1867strife}.  We 
will talk about different ways that this relationship has played out 
in the developed world and the global south
\cite{kenarougluimplications, alcorta1995impact}, and the ways that 
this relationship has evolved over time as the nature of automation 
has changed. This will entail an examination of early reactions against 
automation, including luddite actions carried out in 19th century 
Europe, but will also discuss more modern concerns regarding automated 
checkout machines and more modern industrial automation.  We will also 
touch on anticipated trends\cite{goodall2014ethical} within this realm, 
with the advent of 3D printing and self-driving cars, which proponents 
claim will democratize the means of production and free humans from the 
burden of conveyance, respectively.


\bibliography{citations}
\end{document}
