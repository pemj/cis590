\section{Introduction}
\label{sec:-intro}
"Sweating as he turned here and there to his bellows busily, since he was working on twenty tripods which were to stand against the wall of his strong-founded dwelling. And he had set golden wheels underneath the base of each one so that of their own motion they could wheel into the immortal gathering, and return to his house: a wonder to look at. These were so far finished, but the elaborate ear handles were not yet on. He was forging these, and beating the chains out" writes Homer in Iliad \cite{homer1194iliad}. In fact, Iliad is not alone in ancient literature to describe such contraptions; mentions of automatons can be seen throughout other Greek texts \cite{aaron2015automatones}, ancient Liezi text of China talks about an android singer presented to the fifth king of the Chinese Zhou Dynasty, King Mu of Zhou \cite{lie600liezi}, the 11th/12th century Indian text Lokapannatti talks about robot guards and soldiers \cite{sarah1997lokapannatti}. Even though one might question whether these are about actual incidents or not, these records make one thing certain. That is the fact that the idea of self-working machines is not a new concept to humanity. It is just an extension of the unquenchable human desire to make their lives easier; the desire that gave birth to the rudimentary tools that propelled us up the evolutionary tree, branching up and away from our animal brethren. 

With the digital revolution spearheaded by the discovery of the transistor, we are now at a time when the very same stories, which one might brush off as fantasies when read in the above ancient texts, are slowly but surely becoming commonplace. In place of the auto moving tripods of Hephaistos in Iliad, we now have auto driving cars created by Google \cite{google2016car} and Autopilot mode enabled cars by Tesla Motors \cite{tesla2016car}. Instead of stopping at the creation of singing automatons such as the one from the Liezi text, we have created automated systems that can compose music \cite{liu2014bach} and create artworks \cite{james2016art}. We are using drones and other unmanned vehicles to wage war. Fully automated security robots are emerging \cite{justin2016security}. So it is rational to say that we are going far and beyond the wildest imaginations of our forefathers. 

The discussion about artificial intelligence have grown to the level of debating whether a future artificial intelligence that is comparable to a human in all wakes of life (so that unless specified a human will not be able to distinguish between the said AI and another human\cite{eyal1999turing}), is sapient of not. If we do agree that the AIs are sapient, from that agreement raises the question, "Is humanity using slave labor when it employees sapient robots to do tasks for them?" If a robot commits an offence that is punishable by law, who is liable? Is it the creators of the AI program or is a sapient AI able to take legal responsibility of its own actions? If it is the latter, what kind of rights would the AI have in a legal setting? If it is the former how is it different from the "Sins of the son" fallacy where parents are held liable for crimes committed by their children. 
 
Like every scientific breakthrough before it, automation has created a dichotomy; one group consisting of optimistic scientists such as sir Arthur C. Clarke who claim "The only way of discovering the limits of the possible is to venture a little way past them into the impossible" \cite{arthur1962limits}, and the other consisting of scientists such as Stephen Hawking, Elon Musk, and Bill Gates who are rather pessimistic about the advent of artificial intelligence and thus warns that humanity will have to be diligent as the technology develops \cite{michael2016warning}. Those who are not optimistic about the emergence of the automation also claim that it need not be developed to yield killer robots or artificial general intelligence \cite{luke2013agi} for automation to have a deep social impact. They clame that even a small amount of automation might have far reaching consequences.  The objective of this paper is to analyze that social impact in an ethical standpoint and propose an ethical framework that can be used to mitigate the possible undesirable consequences of automation. 

In section~\ref{sec:-background} we will discuss the socio-ethical impact that the industrial revolution had on the society given that fact that it is the only adequately documented closest approximate technological automation based transformative period of human history that we can compare this new contemporary transformation with smarter and almost autonomous machines. Then we discuss the situation that the society and technology currently stands in section~\ref{sec:-current}. Moving forward, in section~\ref{sec:-other_solutions}  we analyze some of the main solutions that various individuals and groups have proposed to handle the situation. Here we talk about the following solutions; ~\nameref{sec:-singularity},~\nameref{sec:-bill},~\nameref{sec:-jason},~\nameref{sec:-amish}, and~\nameref{sec:-dgr}. For each of these solutions we discuss the pros and then we discuss the shortcomings of these solutions which makes solitary implementation of these existing methods inadequate and sometimes even unethical.

In section~\ref{sec:-our_solution} we will discuss the solution that we propose in this paper. The solution is explained in terms of; ~\nameref{sec:-identify},~\nameref{sec:-general}, and~\nameref{sec:-value}. We explain how out proposed framework can be used by way of the auto-driving car use case in section~\ref{sec:-application}. Finally, the section~\ref{sec:-conc} gives the conclusions of the paper.  

%%% Local Variables:
%%% mode: latex
%%% TeX-master: "main"
%%% End:
