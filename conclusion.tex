\section{Conclusion}
\label{sec:-conc}
Rise of the automated tools designed to accomplish certain tasks and ethical problems stemming thereof is a very potent area for dialogue. It is evident that the set of current solutions have more cons than pros when one put them under scrutiny. We claim that the framework that we proposed in this paper would minimize the short comings and maximize the gains. The efficacy of the proposed solution was evident in the given use case about self-driving cars. We have showed that the proposed solutions would safeguard the autonomy of self, welfare/well-being, and the economic goals of the relevant stakeholderes in such a way the maximum utility is obtained while minimizing the harm. It is worthy to note that, while competent, at a certain point when a certain threshold is exceeded we are still yielding the control to a human. The very existence of this option may help some reluctant individuals to reduce their resistance towards automation because, humans no matter how illogical it is, tend to think that if they have control they can fix any problem despite the fact that an autonomous system deemed a feasible solution does not exist. But one must always remember that until we do obtain a certain level of automation, real the social impact of such a technology can only be speculated. Minute details that escape the most prudent analysis might have very far reaching consequences. And thus we stand on the precipice and stare into the void, and does the void stare back into us?

%%% Local Variables:
%%% mode: latex
%%% TeX-master: "main"
%%% End:
